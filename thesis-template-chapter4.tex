\chapter{Реализация и экспериментальная проверка}


\section{Выбор инструментальных средств}

Если оценивать систему с точки зрения времени выполнения, то самым проблемным участком является большое количество лучей, расчет пересечений которых занимает значительное время. Однако если производить расчеты параллельно, время вычислений уменьшается соответственно количеству одновременно работающих потоков. Лучше всего в настоящее время для реализации параллельных вычислений приспособлены объектно-ориентированные языки программирования.

Дальнейший выбор ограничим современными, широко используемыми языками: С\#, Java, PHP. Данная выборка основана на том, что конечная система будет являться модулем визуализации, а, следовательно должна без особых проблем интегрироваться в множество других программных продуктов. Учитывая, что веб приложения обычно не требуют для своей работы сложных систем визуализации, а тем более алгоритма трассировки лучей, то PHP стоит исключить. Выбор из оставшихся двух весьма затруднителен, так как каждый из них имеет свои преимущества и недостатки. Однако основываясь на том факте, что производительность данного модуля может быть существенно увеличена с помощью видеопроцессоров, позволяющих производить в разы больше параллельных вычислений за единицу времени, а основной инструмент для работы с ними - CUDA (Compute Unified Device Architecture) \cite{Cuda2} использует подмножество языка C, следовательно язык С\# \cite{Troel,Riht} оптимальнее подходит для разработки данной системы.

Как уже было описано, дальнейшая разработка потребует интеграцию работы с видеопроцессором. Для этих целей будет использоваться практически единственная на данный момент доступная и оптимизированная библиотека CUDA.

Необходимость в разработке некоторых структур и функций можно избежать, используя готовую библиотеку для работы с трехмерной графикой Direct3D от компании Microsoft Corporation.

Для тестирования модуля будет использован интерфейс, созданный с помощью .NET Forms. Он позволяет подключить основные функции нашей системы и разносторонне проверить возможности вывода.

\section{Состав и структура реализованного программного обеспечения}

Исходя из того, что данный тип визуализации является на данный момент исследовательским, а также учитывая широкое распространение и проработанность других методов визуализации приложение было разработано как модуль визуализации. Этот модуль может быть интегрирован в множество систем, использующие стандартные методы визуализации, так как алгоритму не требуетсядля работы дополнительных данных. Наиболее эффективно использовать данный модель для клиентских приложений, созданных на языке C\#, из-за легкой интеграции и возможности дополнительного модифицирования.

Для проведения тестирования был создан интерфейс, имитирующий действие реальной системы, обеспечивая алгоритм входными данными и позволяющий осуществлять вывод конечного изображения пользователю для оценки работы алгоритма.

Для работы алгоритма требуется предустановленное программное обеспечение:

\begin{itemize}
	\item Microsoft Visual C++, все пакеты версий 2008-2015
	\item Microsoft XNA Framework Redistributable 4.0
	\item DirectX версии 11 и выше. Работа на низких версиях возможна, но не гарантируется.
	\item Драйвера видеокарты, если таковая установлена
\end{itemize}

Эффективность работы модуля напрямую зависит от параметров системы, несмотря на это модуль может быть запущен на большинстве устройств, удовлетворяющих требованиям. Особо стоит заметить, что существенный рост производительности модуля может быть достигнут за счет улучшения некоторых аппаратных частей: увеличение количества ядер процессора приводит к соразмерному ускорению алгоритма, использование вычислений на видеокарте помогает распределить вычисления и разгрузить основной процессор, тем самым уменьшая время выполнения.



\section{Разработка инструмента альтернативной визуализации}

Необходимость поддержки DirectX, обеспечения совместимости с  С\# и наличие в текущем прототипе библиотеки SharpDX обуславливает разработку инструмента посредством данных технологий. Дополнительных зависимостей для разработки не потребовалось. Исхдя из концепции модульной разработки инструмент получил в результате разработки следубщие возможности:

\begin{itemize}
	\item Визуализация сцены в зависимости от полжения камеры.
	\item Работа с примитивами в виде треугольников,  линий, точек.
	\item Минимальные вершинные и пиксельные шейдеры.
	\item Возможность получения и обработки данных модели частиц.
\end{itemize}

Таким образм поддерживается возможность тестирования новых разработок и их предварительная проверка без необходимости адаптации к алгоритму трассировки лучей. Также благодаря модульной структуре при дальнейшей необходимости расширения функционала инструмента это можно сделать с помощью добавления новых модулей. Пример работы приведен на рисунке (см. \ref{pic:B41})

\begin{figure} 
\begin{center}
\includegraphics[width=.5\columnwidth]{./img/B41.png}
\end{center}
\caption{Демонстрация работы инструмента с системой частиц}
\label{pic:B41}
\end{figure}

\section{Создание алгоритмов интреполяции объектов}

На данном этапе были реализованы алгоритмы разбиения и поиска по разбиению. Для проверки реализации используется разработанный ранее инструмент визуализации.

Результаты работы приведены в приложении \ref{screenshots}

На данном этапе в следствии ограничений по времени не удалось разработать полноценную структуру метода моделирования объектов посредством сплайнов, однако на основе проведенных исследований доказана его применимост к текущей системе, а также логичность использования. Таким образом рациональность разработки данного метода в последующих версиях прототипа оправана и работа над ней будет проведена.

Реализацию алгоритма интерполяции на данном этапе можно представить в следующем виде: (см. \ref{pic:B46})

\begin{figure} 
\begin{center}
\includegraphics[width=.5\columnwidth]{./img/B46.png}
\end{center}
\caption{Диаграмма взаимдействия метода интреполяции объекта}
\label{pic:B46}
\end{figure}

\section{Выводы}

В результате разработки был получен инструмет альтернативной визуализации системы частиц, позволяющий тестировать новые алгоритмы и сравнивать результаты работы с методом визуализации посредством трассировки лучей.

Были реализованы алгоитмы разбиения пространства и поиска по разбиению, которые будут использованы впоследствии для создания нового метода моделирования..

Провеена работа над обеспечением совместимости нового инструмента и текущего прототипа.

Результаты тестирования были сохранены для дальнейшего сравнения с будущими версиями программы. На основе их можно будет судить о улучшении реалистичности изображения в последующих тестах.

Также было проведено сравнение данного модуля с существующими аналогами, выделены преимущества и недостатки, а также добавлен приоритет некоторым задачам для реализации в последующих версиях.