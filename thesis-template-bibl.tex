\phantomsection
%\addcontentsline{toc}{chapter}{\bibname}	% Добавляем список литературы в оглавление
\bibliography{chapters/biblio}						% Подключаем BibTeX-базы

\addcontentsline{toc}{section}{Список литературы}
\begin{thebibliography}{99}


\bibitem{Idef} {Черемных С.В., Семенов И.О., Ручкин В.С. Моделирование и анализ систем. IDEF-технологии. — М.: Финансы и статистика, 2006. — 188 с.}

\bibitem{Troel} {Троелсен Э. Язык программирования C\# 2008  и платформа .Net 3.5  М.: издательство <<Вильямс>>, 2010. 1344 с.}

\bibitem{Riht} {Рихтер Дж. CLR via C\#. Программирование на платформе Microsoft .NET Framework 4.5 на языке C\#./ изд. 4-е. -СПб.: Питер,2013. — 896с.}

\bibitem{Cuda} {Параллельные вычисления на GPU. Архитектура и программная модель CUDA: Учебное пособие. А. В. Боресков и др. Предисл.: В. А. Садовничий Издательство Московского университета, 2012 . — 336 стр.}

\bibitem{Refr} { SCHLICK C. (1994) An Inexpensive BDRF Model for Physically based Rendering, dept-info.labri.u-bordeaux.fr/\%7eschlick/DOC/eur2.html}

\bibitem{Raytr} {Shirley, P., Morley, K. Realistic Ray Tracing. Second edition. Massachusetts: A K Peters, 2003. 239 P.}

\bibitem{KDTree} {Shevtsov M., Soupikov A., Kapustin A. Highly Parallel Fast KD-tree Construction for Interactive Ray Tracing of Dynamic Scenes. In Proceedings of the EUROGRAPHICS conference, 2007.}

\bibitem{Cuda2} {Фролов В, Игнатенко А. Интерактивная трассировка лучей на графическом процессре с применением технологии CUDA. In: GraphiCon'2008. Москва, Россия; 2008. p. 311-2.}

\end{thebibliography}
