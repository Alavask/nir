\chapter{Интерполяция объекта}\label{screenshots}
% как убрать этот пропуск, так же не должно быть!

В данном приложении представлена последовательность обработки обработки объекта алогритмами разбиения и поиска по разбиению.
Результаты оформлены в виде изображений.

\begin{figure}
\begin{center}
\includegraphics[width=.5\columnwidth]{./img/B42.png}
\end{center}
\caption{Загрузка объекта}
\label{pic:B42}
\end{figure}


\begin{figure}
\begin{center}
\includegraphics[width=.5\columnwidth]{./img/B43.png}
\end{center}
\caption{Наложение сетки}
\label{pic:B43}
\end{figure}


\begin{figure}
\begin{center}
\includegraphics[width=.5\columnwidth]{./img/B44.png}
\end{center}
\caption{Получение точек}
\label{pic:B44}
\end{figure}


\begin{figure}
\begin{center}
\includegraphics[width=.5\columnwidth]{./img/B45.png}
\end{center}
\caption{Пересечение лучом}
\label{pic:B45}
\end{figure}
