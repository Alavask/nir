\chapter*{Введение}
\label{sec:afterwords}
\addcontentsline{toc}{chapter}{Введение}

На протяжении обучения в университете среднестатистический студент, согласно действующим указам Министерства образования и учебным программам, выполняет около 10 учебно-исследовательских работ. Учитывая большое количество требований, предъявляемых к оформлению и структуре данных работ, различные критерии, используемые научными руководителями и кафедрами, а также постоянно изменяющиеся Министерством образования выходные характеристики работы, следует признать, что использование стандартных инструментов редактирования текстовых документов, а также средств форматирования, используемых для создания данной работы, становится недостаточно. Разработка единого инструмента для работы с исследовательскими работами студентов позволит не только облегчить данный процесс для студентов и их руководителей, но и позволит увеличить качество работ, автоматизировать многие процессы, позволяя тем самым улучшать качество обучения.

Поскольку в настоящее время для написания работ студентами используются разнообразные инструменты и приложения. Исходя из этого факта конечные документы получаются различными по оформлению и стилистике, что уже является проблемой для проверяющих их сотрудников, но также таит в себе еще одну довольно большую проблему: используемые различными программами форматы файлов и методы их обработки не только не позволяют переводить документ из одного формата в другой без возможной потери некоторых данных, но и образовывают определенные проблемы с безопасностью этих данных. Поскольку многие исследования, проводимые в университете не только обладают уникальностью и определенным уровнем доступа к ним, но и являются собственностью университета, то обеспечение высокого уровня защиты этих данных является одним из приоритетных направлений при разработке данной системы.

В данной работе осуществлена разработка подсистемы управления пользовательскими профилями, а также единого сервиса авторизации для всех компонентов системы.

Работа разделена на четыре раздела: исследование, описание концептов, проектирование и реализация. В каждом из них поставлены и выполнены определенные задачи, соввокупность которых обеспечила правильную разработку системы.

Также следует учесть что в работе приведена лишь тестовая версия подсистемы, что означает возможность наличия некоторых несправностей, связанных с трудностями разработки интегрированных систем. При последующей разработке и приблежении системы к релизной версии, найденный ошибки будут исправлены.
