\chapter{Результаты проектирования дополнительного модуля визуализации}

В данном разделе описываются требования, предъявляемые к разрабатывамому интерфейсу для работы с системой частиц посредством стандартной DirectX реализации. Исследуются необходимые методы и их комбинации для последующего разделения на логические блоки интерфейса. Выявляются основные примеры использования интерфейса и на их основе происходит его планирование.


\section{Требования совместимости}

Одной из целей создания альтернативного метода визуализации является разработка инструмента, позволющего взаимодействовать с текущим прототипом при этом предоставляя разработчику возможности, недступные ранее. Таким образом при проектировании следует учитывать не только добавление новых возможностей, но и организацию совместимости с текущим прототипом системы. Это означает что проектируемый инструмент должен оказывать минимальное воздействие на систему при этом расширяя её возможности.

Для обеспечения совместимости следует определить параметры, с которыми будет взаимодействовать разрабатываемый инструмент:

\begin{itemize}
	\item Соввокупность объектов, представляющая собой систему частиц. Эти данные вляются основой работы и должны передаваться аналогично стандартной реализации. 
	\item Камера и связанные с ней параметры. Так как алгоритм трассировки лучей требует для работы куда больших параметров камеры, чем другие методы визуализации, следовательно в данном случае не требуется передача всех параметров, достаточно лишь положения и направления, аналогично и с дополнительными параметрами конечного изображения, изменение которых стоит перенести на разрабатываемый инструмент для расширения его функциональности.
	\item Метаданные о частицах, дополнительные рассчеты, необходимые для работы трассировки лучей и остальные данные можно не передавать в новый метод, тем самым не нагружаяя его дополнительной обаботкой и предоставляя разработчику возможность самостоятельного расширения.
\end{itemize}

На основе этих пааметров можно определить новую структуру системы и отразить это на диаграмме (см. \ref{pic:B31})

\begin{figure} 
\begin{center}
\includegraphics[width=.5\columnwidth]{./img/B31.png}
\end{center}
\caption{Включение инструмента в систему}
\label{pic:B31}
\end{figure}

\section{Проектирование интерфейса}


Создание интерфейса для для взаимодействия системы цастиц с алгоритмом трассировки лучей необходимо для облегчения работы с ними, отсутствия необходимости изменения кода для внесения модификаций, стандартизации изменений, доступных пользователю и увеличению доступности некоторых возможностей. Также наличие интерфейса позволит проводить дальнейшие исследования в данной системе. 

Проектирование требут опеределенных целей для его проведения:

\begin{itemize}
	\item В зависимости от требуемых результатов работы с системой, пользователя могут требоваться не все доступные возможности, а лишь их ограниченный набор, что позволяет разделить интерфейс на блоки, организованные по принципу совмещения подобных друг другу возможностей вместе. Такой подход позволяет максимально эффективно ипользовать доступные возможности, а также быстро находить необходимые.
	\item Так как многие характеристики, доступные для редактирования пользователю лежат в ограниченных пределах, следует организовать стандартизированные поля для ввода пользовательских данных. Такой подход позволит избежать ошибок при выходе значений за допустимые пределы, а также предоставит пользователю информация о изменяемом значении.
	\item Исходя из того, что эффективность конечного изображения зависит от используемых скриптов, данный тип взаимодействия следует отделить от остальных возможностей и организовать отдельное управление. Это позволит ползователю активировать и отключать доступные скрипты не совершая дополнительных действий.
	\item Конечное изображение может быть различным не только из-за различных настроек камеры, но и из-за применяемых фильтров. Таким образом одновременно могут быть получены несколько видов конечного изображения. Следует добавить возможность разделения основного поля для вывода изображения в зависимости от настроек. Такой подход позволит изучать систему частиц более подробно без необходимости изменения настроек вывода, также это дает доступ к просмотру сцены с разных точек, что в свою очередь увеличивает количество получаемой из неё информации.
\end{itemize}

На основе данного прототипа можно построить интерфейс.

\section{Выводы}

В результате проектирования была получена модель инструмента альтернативной визуализации, позволяющей работу с системой частиц без использования алгоритма трассировки лучей. Это позволяет обеспечивать облегченную работу с системой частиц, а также возможность тестирования некоторых алгоритмов и методов без необходимости адаптации к системе трассировки.

Результаты проектирования оформленны в виде диаграмм.

Были спроектированны:

\begin{itemize}
	\item Основные логические блоки системы и определено взаимодействие между ними.
	\item Новый метод взаимосвязей компонентов системы и их данных.
	\item Связи, основанные на предъявляемых к системе требованиях. Их определение построено на сценариях работы с системой и её возможностях.
	\item Прототип инструмента альтернативной визуализации.
\end{itemize}