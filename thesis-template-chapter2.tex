 \chapter{Разработка методологии разбиения и моделирования трехмерных объектов}

Данный раздел логически разделен на две части: разработка алгоритмов разбиения и поиска по созданному разбиению, а также разработка моделирования объектов на основе разбиения.

\section{Методы разделения трехмерного пространства}

В этой главе приводится метод разбиения пространства, необходимый для упорядоченного хранения данных об объекте, выбор основания разбиения будет произведен на основании удобства испоьзования полученного разбиения для последующего слайнового моделирования, подробнее о котором рассказано во второй части этого раздела.

Исходя из того, что конечное разбиение объекта должно содержать упорядоченный набор точек, а также значения связности между ними можно предложить следующие алгоритмы разбиения:

\begin{itemize}
	\item Покрытие объекта точками с ограничением по максимально заданному расстоянию. Выполняется с помощью рекрсивного перебора возможных значений для создания новой точки. Из-за рекурсивности алгоритм сильно ограничен размерами применения а также не гарантирует точное выделение границ объекта.
	\item  Использование обратного алгоритма выпуклой оболочки, заключающегося в поиске максимальных возможных расстояний в множественных сечениях объекта плоскостями и последующих сохранением точек максимально удаленных друг от друга. В отличии от предыдущего алгоритм гарантирует  сохранение границ разбиваемого объекта, однако выбор плоскостей ля разбиения сильно зависит от конкретного объекта, что приносит очередную алгоритмически сложную задачу в виде поиска этих плоскостей.
	\item  Оснвываясь на приведенных алгоритмах можно заметить отсутствие в них фиксированных компонентов, которые могли бы уменьшить сложность вычислений за счет использования дополнительной памяти. Для трехмерного пространства таким компонентом может служить сетка, зафиксированная в пространстве, при этом для дополнительной оптимизации можно использовать её вариант с перпендикулярными плоскостями сечения и фиксированным размером ячейки.
\end{itemize} 

Рассмотрим поподробнее последний алгоритм. Его работу можно проиллюстрировать на рисунке (\ref{pic:B21}). Для большего понимания этот и дальнейшие примеры будут представлены в виде двумерных объектов, как частичный случай работы.

Итак, имея синий объект произвольной формы и наложив на него заранее сгенерированную сетку можно определить места пересечения поверхности объекта с этой сеткой. В данном случае все пересечения отмечены красным. Далее все ячейки сетки, которые содержат 2 и более пересечения на границах необходимо выделить отдельно и вынести в список, так как наличие более 2 пересечений свидетельствует о том, что поверхность проходит через эту ячейку. Отметим такие ячейки зеленым. После этого поведение поверхности оказывается определено на уровне ячеек, однако внутри данных ячеек поверхность не обязательно ведет себя как примитив, построенный по граничным точкам. Для описания таких случаев следует опеделить некий алгоритм поиска выступающик точек как максимумом или минимумов, и аналогично точкам пересечения сохранить информацию о этих точках внутри ячеек.

\begin{figure} 
\begin{center}
\includegraphics[width=.5\columnwidth]{./img/B21.png}
\end{center}
\caption{Алгоритм разбиения}
\label{pic:B21}
\end{figure}

В сумме на выходе данного алгоритма получается следующая структура: 

\begin{itemize}
	\item Набор ячеек (зеленых), хранящих в себе часть поверхности.
	\item Набор точек пересечения с сеткой (красных), хранящихся внутри ячеек, при этом точка находящаяся на границе принадлежит одновременно обоим ячейкам.
	\item Набор точек внутри ячеек (красных), также хранящийся внутри ячеек и  представляющий собой отклонения поверхности от стандартизированного полигона на размере ячейки.
\end{itemize} 

\section{Поиск по разбиению}

Получив на выходе разбиение объекта дальнейшее его использование требует систему, осуществляющую поиск по списку ячеек с целью нахождения пересечения луча и ячейки, а также поиск пересечений луча с поверхностью уже внутри ячейки. 

Как и любой другой поиск основными целями данного алгоритма являются: 

\begin{itemize}
	\item Минимальное время нахождения требуемого объекта.
	\item Рассмотрение всех возможных вариантов без пропуска даже маловероятных.
	\item Обеспечение максимального отсечения ячеек, вероятность пересечения с которыми равно нулю. 
\end{itemize} 

Основываясь на требованиях разумно заложить в основу данного алгоритма уже существующий алгоритм Брезенхэма аппроксимации отрезка на дискретном поле. Так как луч в  системе является прямым независимо от других обстоятельств, это упрощает последующую работу, а следственно модификации алгоритма для остроения кривых приведены не будут. Пример работы алгоритма представлен на рисунке  (\ref{pic:B22}).

\begin{figure} 
\begin{center}
\includegraphics[width=.5\columnwidth]{./img/B22.png}
\end{center}
\caption{Пример}
\label{pic:B22}
\end{figure}

Луч (синий), пропускаемый сквозь сетку, аналогичную той что была использована для разбиения, пересекает её ячейки. Алгоритм, зная размер сетки выбирает одну из координат как основную, рассчитывает положение центральной точки отрезка в каждой ячейке используя основную координату. Используя получившееся значение выбирается конкретная ячейка в которой лежит центральная точка отрезка и её положение сохраняется. Такжим образом получается последовательность ячеек (красных), соединенных углами, через которые проходит начальный луч.

Однако такой подход обеспечивает лишь непрерывную последовательность, не гарантируя что все точки луча будут находитьс в выделенных ячейках. В таком случае следует модифицировать алгоритм, добавив к результату окресность каждой ячейки, в виде соседних ячеек, соединенных по граням (зеленые). Такой подход гарантирует попадание всех участков луча в определенную ячейку.

Также стоит обратить внимание на то, что данный метод имеет дополнительные преимущества при работе с принятым методом разбиения: так как положение каждой ячейки сетки определяется координтами и заранее фиксированно, то поиск по ячейкам, которых нет в списке результата алгоритма разбиения, занимает минимальное время так как обеспечивается прямая ссылка и простая проверка. Также стоит учесть направленность луча, что позволяет выводить результат поиска как направленную последовательность ячеек, что в свою очередь позволяет отбросить проверку части ячеек, если пересечение было найдено ранее в последовательности.

\section{Моделирование конечного объекта}

Основываясь на анализе, приведенном в предыдущем разделе, а также возможностях разработки был выбрн метод сплайнового моделирования для модификации прототипа в силу следующих причин:

\begin{itemize}
	\item Дополнительная разработка алгоритмов не требует больших затрат.
	\item Данный метод имеет доступную информацию для изучения и разработки.
	\item Использование этой методологии позволит раскрыть взаимодействие системы частиц и алгоритма трассировки лучей за счет создания более гладких, а следовательно реалистичных моделей, что отлично совместимо с трассировкой лучей и при этом может быть легко связанно с системой частиц.
\end{itemize} 

Применение сплайнов требует для создания модели набора точек, по которым будут строится сплайны, что было полученно в предыдущих главах. Имея набор точек, распределенный по ячейкам последующая их обработка в сплайны состоит из этапов:

Для начала требуется определить на основе каких точек будет строится сплайн. Если брать для рассчета только точки внутри ячейки - конечный объект будет похож на аналогичный, построенный с помощью примитивов, при этом некоторые части мповерхности могут отсутствовать. Таим образом для создания сплайна требуется получить не только набор точек, лежащий в ячейке пространства, но и определить соседние с ней ячейки и продолжить сплайн, опираясь расширенное множество точек. При этом стоит учитывать что граничные точки хранятся в нескольких ячейках, а следовательно их необходимо учитывать лишь один раз.

Если после получения первичных сплайнов реализовать вывод модели, то с большой вероятностью она будет лишь частично напоминать исходный объект. Проблема заключается в том, что сплайны, построенные на основе различных ячеек могут не совпадать друг с другом из-за метода их построения. В таком случае следует произвести дополнительное вычисление граничных случаев для организации совместимости границ сплайнов, тем самым устраняя разрывы в объекте.

Так как полученые сплайны являются лишь кривыми, описывающими общие границыфигуры, то пространство между сплайнами ничем не заполнено. Именно поэтому на последнем шаге и вычисляется поврехность между сплайнами, на основе приведения к барицентрическим координатам и использованию сплайнов как координатных осей. Вычисления для каждой точки не требуются, так как данный участок выцчисляется лишь при прямоом запросе, полученным от алгоритма поиска пересечения, при пересечения луча с конкретной ячейкой, тем самым позволяя увеличивать производительность. Также такой метд получения поверхности позволяет получить нормаль к поверхности в точке, что в свою очередь необходимо для получения отраженного и преломленного лучей. Это позволяет испльзовать всю мощность визуализации посредством алгоритма трассировки лучей при этом позволяя создавать куда более сложные модели, а также генерировать их динамически посредством связи с системой частиц. 


\begin{figure} 
\begin{center}
\includegraphics[width=.5\columnwidth]{./img/B23.png}
\end{center}
\caption{Трехмерный пример сплайнового моделирования}
\label{pic:B23}
\end{figure}

Разберем принцип на основе примера (\ref{pic:B23}).

В примере выделены 3 ячейки, содержащие точки (красные). Для начала будут построены отдельные сплайны для каждой ячейки. После чего они будут объединены в общие сплайны (синие). После того при получении запроса на пересечение ячейка, через которую проходит луч обрабатывает сплайны в ней и на основе положения луча и сплайнов вычисляет точку пересечения или же выдает сообщение что пересечения не обнаружено.
\section{Выводы}

В процессе исследования были определены основынеалгоритмы разбиения поиска и моделирования трехмерной поверхности.

Были проанализированны методы разбиения пространства, выбран оптимальный для прототипа, была показана его реализация, а также возмжности и ошибки, которые могут возникнуть при работе с ним.

Помимо этого изучены методы поиска, основанные на созданных ранее методах разбиения с учетом типа модели и характеристиками алгоритма трассировки. Был предложен алгоритм поиска существенно сокращающий время на выполнение, за счет прямого доступа к элементам, а также отбрасывания ножества значений, являющихся лишними.

Описан использование нового метода моделирования - сплайнового алгоритма, позволяющего задавать более плавные объекты, а также реализующего возможность динамического создания объектов сложных форм, приведен пример его использования.

Используя совокупность описанных методов и предлагаемых возможностей по улучшению взаимодействия системы частиц и алгоритма трассировки лучей, стоит сделать вывод о рациональости интеграции интеграции альтернативных методов визуализации, так как использование позволяет расширить возможности по представлению конечного изображения данной системы.