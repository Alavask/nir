\chapter{Анализ систем, использующих пользовательские профили и методы обеспечения их безопасности}
\label{chapter1}

В данном разделе приводятся основные понятия и специфика проблемной облести, анализируются аналогичные технологии и задается основное направление исследования. Основной целью является реализация безопасных методов аутентификации и использования системы для генерации и редактирования учебно-исследовательских работ.

Проведя анализ технической литературы был составлен список функционала, необходимый данной системе для обеспечения максимальной эффективности работы в ней.

На основе изученных данных в конце раздела была определенна цель исследования,  которую необходимо достигнуть в рамках данной учебно-
исследовательской работы. Также были выделенны подзадачи, выполнение которых позволяет достигнуть поставленной цели.


\section{Обзор доступных технологий для работы с документами}

Создание учебно-исследовательской работы состоит из множества этапов: анлиз проблемной области, выделение актуальных проблем, исследование метдов их решения, создание прототипов и готовых систем. В зависимости от сферы исследований и задаваемых требований этот список может изменяться, количество необходимых этапов может как увеличиваться так и уменьшаться, иногда даже невозможно заранее оценить все необхдимые этапы из-за высокой сложности исследований. Казалось бы, в общем случае невзможно создать алгоритм, способный определять полную последовательность действий, однако как минимум один компонент, общий для всех исследований всегда присутствует - это результат данного исследования, выраженный в виде документа.

Документ, получаемый в результате работы может быть распространен в неком сообществе для ознакомления и использования его для дальнейших работ, также документ подтверждает наличие результата и позволяет суммировать, сохранить и передать другим плученную информацию. Таким образом создание данного документа не только необходимый при проведении исследования этап, но довольно важный аспект всей работы, что опеределяет требования к его созданию. Поскольку разрабатываемая тестовая система в данный момент рассчитана только на создание документов, соответствующих профилю кафедры, различия в ограничениях на создания данных документов здесь и далее будут фиксированы внутреними уставами кафедры и возможность их изменения предусмотрена не будет, вместо этого, основной целью будет функционал финальной системы.

В общем случае создание документов подразумевает работу со следующими компонентами:

\begin{itemize} 
	\item Форматирование всех остальных блоков: взаимное расположение, структура документа, расположение на странице, ссылки и управление, списки и другие перечисления, статические блоки и тд.
	\item Текстовые объекты с разнообразными возможностями отображения: изменение размера, выделение цветом, различные стили и шрифты, скрытые символы управления.
	\item Вывод и упорядочивание большого количества информации: таблицы, базы данных, графики.
	\item Формулы и другие виды информации о проблемной области.
	\item Библиографическая информация о использованной литературе и её характеристиках
	\item Блоки кода, с возможностью отображения синтаксисов и проверки языковых конструкций.
	\item Изображения, соответствующие данному исследованию или иллюстрирующие некоторые процессы.
	\item Схемы, описывающие определенные явления в проблемной области и упрощающие восприятие документа.
\end{itemize}

Поскольку при создании документов используются вышеперечисленные компоненты в различном соотношении, то в зависимости от предпочтений и требуемого функционала используется достаточно большой и разнообразный набор программного обеспечения. В теории возможно объединить весь необходимый функционал в одно приложение, однако на практике приложение либо получится довольно большим и множество его функций не будут использоваться, либо наоборот, пользователям будет не хватать некоторых функций. Также при использовании такого подхода будт использоваться вычислительные ресурсы устройств, что тоже может стать проблемой.

Исходя из этого следует проанализировать существующие системы, в которых на данный момент создаются документы для учебно-исследовательских работ с учетом их конструктивных возможностей, а также накладываемых ограничений.

Начнем, пожалуй, с самого распространенного ПО для редактирования документов - Microsoft Office Word. Имея широкое распространение среди пользовательсих систем данное ПО приобрело высокую популярность и для создания исследовательских документов. Разнообразный функционал, интеграция под множество платформ и распространенность быстро вытеснить более простые редакторы текстовых документов.Однако за внешним удобством данного ПО скрывается определенное количество проблем, которые, в свою очередь, затрудняют создание документов. Во-первых, межплатформенная интергация продукта получила большую зависимость от конкретных систем, а это приводит к тому, что один и тот же документ, открытый на разных устройствах может иметь разное форматирование и выглядеть по-разному, из-за чего передача  документов может стать затруднительной ввиду наличия дополнительной работы по проверке идентичности. Во-вторых из-за наличия сложных правил форматирования документа многие простые действия пользователя могут привести к нелогичным результатам и время, затраченное на поиск правильного подхода к искомому действию зачастую может занимать довольно большое количество времени. В-третьих работа с документами на пользовательских устройствах не только использует их вычислительые ресурсы, но и зачастую из-за невнимательности пользователя может создать утечку данных из-за которой пользователи , не имеющие доступа к документу, могут получить его.

Далее рассмотрим LaTeX-системы для создания документации. Поскольку данные системы требуют специфических знаний от пользователей для работы с ними, поэтому данный вид систем получил распространение лишь в ограниченных кругах и в общем случае неизвестен обычному пользователю. Однако сложности, связанные с необходимостью использования редактирования текста посредством скриптоподобных конструкций и дополнительных инструментов для сборки конечного документа, могут быть решены посредством разработки визуальных редакторов, позволяющих автоматизировать процесс работы с документами с помощью различных инструментов и надстроек. Существенный минус использования визуальных редакторов состоит в том, что требования, предъявляемые к различным типам документов различны, и создание визуального редактора, позволяющего работать с различными типами документов, скорее всего приведет к созданию масштабного инструментария, не только сложного для понимания конечным пользователем, но и не используемого большую часть времени. Однако разработка визуального редактора для LaTeX-систем преследующих ограниченное число целей, а также лимитированных строгими правилами вполне возможна и может быть реализованна при необходимости.

Следующие на очереди - онлайн системы для создания документов (Google docs, Word online и другие.). С позиции функциональности работы с документами данные системы практически полностью аналогичны их оффлайн аналогам, предоставляя все те же возможности для редактирования при аналогичном визуальном оформлении. Существенным отличием является непосредственное расположение данного ПО и связанные с ним особенности. Поскольку онлайн системы располагаются на удаленных серверах в сети интернет, то их использование не требует занятия ресурсов компьютера и не привязывается к характеристикам конкретной системы. Единственное требование к их использованию - наличие доступа к сети интернет, что одновременно позволяет расширить количество рабочих платформ (в том числе смртфоны и иные устройства), но при этогм создает затруднение при получении доступа к файлу при отсутствующем соединении с сетью. Также использование сетевой модели позволяет синхронизировать файл между несколькими устройствами, тем самым защищая важные данные от потери, а также позволяет обеспечивать одновременную работу над одним и тем же файлом для нескольких пользователей. Существенный недостаток в данном случае заключается в безопасности использования таких систем. Несмотря на высокую защищенность данных систем, их владельцы, а также их подрядчики, работающие над такими системами, все равно могут иметь неограниченный доступ к пользовательским данным и использовать их в личных целях. Такой подход накладывает существенные ограничения на возможности использования данных систем при работе с секретными или защищенными документами.

Рассмотренные инструменты обладают одним общим свойством: поскольку их функционал позволяет работать над многими видами файлов для различных нужд пользователей,  использование их для однотипных повторяющихся задач не реализует присутствующий функционал полостью, а также зачастую затрудняет работу с системой. Для целей, определенных в данной работе, стоит сосредоточится на отдельных аспектах и определить наиболее подходящую систему для работы с документами или же воссоздать её самим. Поскольку среди всех вышеперечисленных систем не нашлось ни одной, недостатками которой можно было бы пренебречь, конечная система, реализуемая в данной работе, будет создана отдельно.

Логично, что создание полноценной системы с нуля займет довольно большое время не только на реализацию, но и на разработку концепции. Для уменьшения временных и иных затрат на создание архитектуры конечная система будет включать в себя некоторые части, подобные другим системам, поскольку эти части (или модули) польностью соответствуют поставленым задачам, а также имеют не только известную, но и протестированную многими пользователями реализацию. Таким образом система будет представлять собой веб-прложение,  представляющее собой визуальный редактор для LaTeX-системы, настроенный для выполнения специфичных операций. 

Поскольку конечная система представляет собой соввокупность приведенных выше систем, помимо их сильных сторон, она также может содержать в себе и их недостатки. Однако именно над их устранением стоит задумываться при создании архитектуры системы. Одним из сложных моментов при сооздании системы является ее безопасность. При неправильном планировании и реализации подходов к осуществлению безопасности дальнейшее создание системы может стать практически бессмысленым, поскольку оданные в любой момент могут быть скомпрометированы. Поэтому подходы к обеспечнию безопасности особенно актуальны при создании конечной системы, а данная работа будет направлена на их реализацию. 

\section{Анализ методов обеспечения безопасности веб-приложений}

Сфера безопасности веб приложений в настоящее время претерпевает постоянные изменения. Новые методы защиты информации появляются все чаще и чаще, так как появление новых уязвимостей требует немедленного реагирования и внесения изменений в существующие продукты. Для реализации безопасной работы с разрабатываемым предложением следует рассмотреть актуальные на момент создания системы уязвимости и предусмотреть возможные решения.

Работа любого веб-приложения всегда неразделима с передачей данных между клиентом и сервером. Однако передача данных напрямую, без использования каких-либо методов их защиты, даже несмотря на использование стандартных сетевых протоколов модели TCP/IP, не обеспечивает их полную сохранность. Поскольку данные проходят через сеть, существует возможность проникновения злоумышленника в один из сегментов этой сети, а следовательно, получение пользовательских данных и дальнейшая подмена или несанкционированное использование. Но если утечка данных все-таки произошла, но злоумышленник получает лишь некий набор символов вместо искомых данных, то и вредоносные действия не могут быть им совершены. Именно так и работают механизмы шифрации: они позволяют создать некий ключ, используя который данные между клиентом и сервером могут быть переданы в закодированном виде, не предоставляющем никакой информации о данных пока ключ отсутствует. Также данный подход позволяет обеспесчивать аутентификацию конкретного пользователя и обеспечивает целостность передаваемого сообщения. Одним из примеров использования данной функции является протокол HTTPS, который помимо передачи гипертекстовых страниц также реализует протокол SSL (\ref{pic:C11}), основанный на криптографическом подходе. На данный момент этот протокол безопасности широко распространен и не требует больших усилий для его реализации и последующей настройки.

\begin{figure}

\begin{center}

\includegraphics[width=.5\columnwidth]{./img/C11.png}

\end{center}

\caption{SSL}

\label{pic:C11}

\end{figure}

Используя шифрование, данные достигают сервера и для дальнейшего их использования и обработки требуется некторый серверый код и именно в нем может существовать следующая уязвимость. Поскольку при непосредственной обрабоке данных внутри кода их защита регулируется лишь техническими ограничениями системы (ограничения типов, функций, выражений) и данные представлены в незашифрованном виде для осуществления операций над ними, то любая часть исполняемого кода, имеющего доступ к ним должна быть заранее проверена не только на осутствие ошибок, но и на возможность случайной утечки этих данных. Так как стандартом разработки считается создание кода с использованием неких методологий разработки и использование строгих правил оформления кода, а также множественную проверку результата другими разработчиками и тестами, то в общем случае исполняемый код разработанного приложения можно считать безопасным. Однако любое вмешательство со стороны даже для небольшого участка кода (\ref{pic:C12}) может превратить приложение не только в небезопасное, но даже во вредоносное. Соответственно защита от внедрения в приложение чужого кода, а также от использования непроверенных сторонних библиотек и сервисов должна быть продумана заранее. Это может быть реализовано с помощью использования статических анализаторов кода, проверенных библиотек и

блокировок на выполнение чужого кода.

\begin{figure}

\begin{center}

\includegraphics[width=.5\columnwidth]{./img/C12.png}

\end{center}

\caption{Инъенкция кода}

\label{pic:C12}

\end{figure}

После обработки сервером данных необходимо обеспечить их дальнейшее хранение и доступ. Самым распространенным инструментом для хранения данных является СУБД, с которой в свою очередь также связаны некоторые проблемы с безопасностью. Самой серьезной на данной момент являются SQL-инъекции (\ref{pic:C13}). Используя доступные для передачи данных методы работы с веб-приложением злоумышленник может добавить в набор данных SQL-код, который может как повредить записи в базе данных, так и изменить саму структуру базы данных, а также получить изначально недоступные для него данные. Несмотря на довольно серьезные последствия, которые могут быть результатом такой атаки, защититься от неё довольно просто: стоит лишь добавить дополнительные проверки на некоторые последовательности символов для входных данных, а также желательно не использовать общеизвестные структуры баз данных во избежание случайных совпадений.

\begin{figure}

\begin{center}

\includegraphics[width=.5\columnwidth]{./img/C13.png}

\end{center}

\caption{SQL-иньекция}

\label{pic:C13}

\end{figure}

Имея данные, хранящиеся на сервере, естественно появляется необходимость в передачи некоторой части этих данных пользователю по его запросу. В этом случае необходимо убедиться, что конкретному пользователю могут быть переданы лишь те данные, к которым он имеет доступ. Это приводит к необходимости создания системы контроля доступа, то есть определенной функции, которая обеспечит разграничение прав пользователей над отдельными файлами и ограничит несанкционированный доступ. Такие системы используют пользовательские учетные записи, за которыми закреплен набор данных, доступных для конкретного пользователя, а также методы доступа к этим записям, включающие в себя авторизацию, аутентификацию и регистрацию. Подробнее они будут рассмотрены в следующей части.

Таким образом основным объектом защиты становятся пользовательские данные во всех их формах, а расположение требуемых для создания защит может быть представленно в виде карты (\ref{pic:C14}).

\begin{figure}

\begin{center}

\includegraphics[width=.5\columnwidth]{./img/C14.png}

\end{center}

\caption{Карта обеспечения безопасности}

\label{pic:C14}

\end{figure}

\section{Методы авторизации в приложении}

Как было выяснено в предыдущей части, для безопасной работы веб приложения требуется метод авторизации пользователей. Для начала стоит определить разницу между терминами идентификация, аутентификация и авторизация в компьютерных системах:

\begin{itemize}

\item Идентификация - получение данных о записи, связанной с конкретным пользователем, посредством предоставления уникального идентификатора для данной записи. Это может быть как номер телефона, адрес электронной почты, уникальное название и тд.

\item Аутентификация - предоставление доказательств, что пользователь запрашивающий доступ к текущей учетной записи имеет права на доступ к ней. Зачастую этой проверкой является ввод пароля от учетной записи, но далее будут рассмотрены иные методы.

\item Авторизация - проверка прав связанной с пользователем учетной записи на доступ к запрашиваемому ресурсу. Например, учетная запись администратора имеет доступ к защищенным настройкам приложения, когда обычные пользователи не имеют прав на доступ к ним.

\end{itemize}

Поскольку для обеспечения безопасности моугт быть использованы различные методы, то следует рассмотреть их и выбрать наиболее подходящий.

Самый стандартный и широко распространенный метод аутентификации основывается на использовании уникального идентификатора для записи и пароля. Он реализован во многих сетевых протоколах, и в том числе - HTTP. В нем после обращения неавторизованного пользователя к защищенному ресурсу пользователю возвращается схема параметров аутентификации, использовав которую для дальнейших запросов с заголовком Authorization пользователь может получать доступные ему данные (\ref{pic:C15}). Аналогична схема с использованием формы аутентификации, высылаемой сервером пользователю, в которой он вводит свои пользовательские данные и отправляет их POST запросом. При успехе сервер возвращает токен сессии, с помощью которого осуществляется последующий доступ.

В общем случае аутентификация на основе пароля не считается надежным способом, поскольку утеря, подбор пароля а также простые ошибки в реализации могут привести к утечке данных. При этом данная реализация не поддерживает сообщения о угрозах, а следовательно брешь в защите может существовать достаточно долго. Также при дальнейшем изучении оказывается что данный способ содержит большое количество возможных угроз безопасности.

\begin{figure}

\begin{center}

\includegraphics[width=.5\columnwidth]{./img/C15.png}

\end{center}

\caption{HTTP аутентификация}

\label{pic:C15}

\end{figure}

Следующий способ - использование сертификатов. Сертификат представляет собой набор атрибутов, идентифицирующих владельца, подписанный certificate authority (CA). В данном случае приложение доверяет CA, который, в свою очередь, гарантирует подлинность сертификатов (\ref{pic:C16}). Также сертификат связан с закрытым ключом, хранящимся у владельца и подтверждающий факт владения. Такая аутентификация используется как часть протокола SSL и широко поддерживается браузерами.

\begin{figure}

\begin{center}

\includegraphics[width=.5\columnwidth]{./img/C16.png}

\end{center}

\caption{Использование сертификата}

\label{pic:C16}

\end{figure}

Во время аутентификации сервер проверяет сертификат используя следующие правила: сертификат подписан CA, сертификат действителен на дату проверки, сертификат не был отозван. После аутентификации на основании данных сертификата производится авторизация, используя информацию, хранящуюся в нем.

Такой подход позволяет существенно усилить безопасность системы, уменьшить количество взломов за счет усложнения системы и позволяет получить сведения о возможных атаках на систему. Однако необходимость распространения сертификатов и трудности их доставки и обслуживания делают данный способ труднодоступным для широкой аудитории.

Использование одноразовых паролей тоже может быть реализовано для целей аутентификации. В основном данный подход используется совместно со вводом основного пароля: пользователь имеет при себе устройство, способное генерировать одноразованые пароли (или же получать их методов, отличным от запрашиваемого), при необходимости аутентификации пользователь вводит сначала собственный пароль, после чего происходит генерация временного пароля, действующего ограниченое время. После его ввода, авторизация завершается.

Одноразовые ключи направлены для быстрой и эффективной проверки действий пользователя, таких как банковские переводы и отправка защищенных данных, но при постоянном использовании в веб-приложении данный подход может быть излишним поскольку пользователю придется подтверждать каждое свое действие, даже если оно не требует защиты.

Таким же ограниченным для отдельных целей способом является аутентификация по ключу: для доступа к ресурсу создается ключ и привязывается к конкретному устройству. В дальнейшем для получения доступа требуется лишь передача данного ключа, что исключает его подбор и позволяет быстро заменить на другой ключ, а также разделить его на секретную и публичную части для улучшения защиты. Этот метод широко распространен в работе с хранилищами, но для веб приложений аналогичен сертификатам по сложности реализации.

Последний способ, набирающий распространение в данный момент является аутентификация посредством токенов. Такой способ применяется для распределенных систем, где функцию аутентификации пользователя можно делигировать другому приложению, отличному от того, к которому идет изначальное обращение. Общая концепция заключается в том, что сервис аутентификации предоставляет веб-приложению данные о пользователе в виде токена, а само приложение использует этот токен для идентификации и авторизации пользователя (\ref{pic:C17}).

\begin{figure}

\begin{center}

\includegraphics[width=.5\columnwidth]{./img/C17.png}

\end{center}

\caption{Использование токена}

\label{pic:C17}

\end{figure}

Для веб-приложений используется перенаправление пользователя на старницу авторизации сервера аутентификации, и обратного перенапрвления, но уже с действующим токеном.

Токен представляет собой структуру данных, содержащую информацию и создателе, получателе, сведения о пользователе и срок его действия. Обычно токены делятся на кратковременные и долговременные. Так, при обращении к серверу аутентификации пользователь получает кратковременный одноразовый токен, позволяющий авторизоваться в веб приложении, а также долговременный токен, позволяющий получать новые кратковременные токены при необходимости. При аутентификации с помощью токена проверяется его предназначение к данному приложению, время действия и его подлинность. В случае успешной проверке приложение выполняет авторизацию на основе данных токена.

Такой способ позволяет не хранить в веб-приложении учетные данные пользователей, а предоставить их защиту другой системе (серверу аутентификации). В последнее время данный способ получает широкое распространение за счет сервисов компний Google, Facebook и других, предлагающих механизмы аутентификации по тоекну с использованием их учетных записей. Также этот подход позволяет отказаться от создания дублирующихся учетных записей в разных системах, а также позволяет уменьшить проблемы от утраты или взлома паролей.

Подводя итог, использовние дополнительных средств помимо паролей увеличит безопасность системы. Таким образом в данной работе из-за особенностей структуры ит-сферы НИЯУ МИФИ будет представлена реализация раелизация как посредством аутентификации, так и с использованием паролей через SSL шифрование.

\section{Наполнение профиля пользователя}

Создавая личный кабинет для пользователя, где он бы мог узнавать необходимую ему информацию о его учетной записи, а также получать доступ к персонализированным функциям, нужно продумать множество аспектов: от удобства пользователя при взаимодействии с элементами до безопасности его данных.

Для начала стоит определить цели, которые хочет достигнуть пользователь, заходя в личный кабинет:

\begin{itemize}

\item Получение актуальной информации о текущем статусе его учетной записи.

\item Возможности, которые в данный момент доступны для его учетной записи.

\item Информация о ближайших событиях и соответствующих им действиях.

\item Функционал доступа к созданным им документам и данным.

\item Получать уведомления о состоянии системы.

\end{itemize}

Исходя из основ разработки веб-приложений у пользователя долен представлять личный кабинет как центр управления его учетной записью и всеми действиями, производимыми в приложении. Это позволит пользователя сосредоточиться на необходимой ему работе уделяя меньше времени на переходы между страницами и поиск необходимых действий.

Для некоторых аспектов следует пренебрегать удобством для улучшения безопасности, например: при регистрации определить необходимый набор полей, убрав лишнее, но при этом оставить все поля обязательными для заполнения, что позволит системе хранить более актуальную и точную информацию о пользователе.

Но также стоит задуматься о удобстве пользователя в некоторых аспектах, зачастую незаметных даже для опытных разработчиков: производить автоматическую аутентификаию после регистрации, "угадывать" расхождение в раскладке пользователя и предлагать правильные варианты, оставить возможность сохранения паролей, отделить концепцию аутентификации от личного кабинета.

Также стоит уделить внимание следующим реализациям:

\begin{itemize}

\item Создание подсказок и логических проверок для заполнения форм.

\item Частое обновление пользовательской информации при её зависимости от другой системы.

\item Возможности отмены совершенных действий

\end{itemize}

Таким образом создание личного кабинета пользователя представляет собой довольно сложный процесс, поскольку включает в себя не только требования к реализации исполняемого кода, но и требования, основанные на пользовательском опыте.

\section{Выводы}

Основываясь на предыдущих разделах, можно составить следующие выводы.

\begin{enumerate}

\item Выполнен анализ доступных технологий для работы с документами, выявлены их преимущества и недостатки.

\item Определены основные блоки, необходимые для создания в разрабатываемой системе, рассмотрены возможные проблемы при их реализации.

\item Проанализированы методы обеспечения безопасности для веб приложений, современные технологии для защиты данных и перечислены основные ошибки.

\item Описаны методы аутентификации пользователей в веб приложениях и построена модель аутентификации, подходящая разрабатываемой системе.

\item Приведено наполнение профиля пользователя, необходимое для работы с системой.

\end{enumerate}

\section{Постановка задачи курсового проекта}

На основе анализа, проведенного в предыдущих подразделах, возможно сформулировать цель, которая должна быть достигнута в рамках данной учебно-исследовательской работы. Необходимо также выделить основные задачи, которые необходимо выполнить для достижения поставленной цели. Целью данной учебно-исследовательской работы является реализация безопасных методов аутентификации и использования системы для генерации и редактирования учебно-исследовательских работ.

Помимо выполнения основной цели необходимо также уделить внимание следующим задачам:

\begin{itemize}

\item Рассмотреть возможные нарушения безопасности веб-приложения и предложить решения.

\item Создать функционал для авторизации как посредством токенов, так и с использованием паролей.

\item Реализовать структуру БД для хранения пользовательских данных.

\item Организовать личный кабинет для отображения пользовательских данных.

\item Провести анализ возможных расширений для улучшения безопасности системы.

\end{itemize}

Выполнение этих задач не обязательно, но желательно для актуализации данной системы.