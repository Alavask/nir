\chapter{Анализ систем, использующих пользовательские профили и методы обеспечения их безопасности}
\label{chapter1}

В данном разделе приводятся основные понятия и специфика проблемной облести, анализируются аналогичные технологии и задается основное направление исследования. Основной целью исследование пользовательского опыта в сфере разработки пользовательских профилей, а также изучение возможностей по создению безопасной среды при работе с разработываемой системой.

Проведя анализ технической литературы был составлен список функционала, необходимый данной системе для обеспечения максимальной эффективности работы в ней.

На основе изученных данных в конце раздела была определенна цель исследования,  которую необходимо достигнуть в рамках данной учебно-
исследовательской работы. Также были выделенны подзадачи, выполнение которых позволяет достигнуть поставленной цели.


\section{Обзор доступных технологий для работы с документами}

Создание учебно-исследовательской работы состоит из множества этапов: анлиз проблемной области, выделение актуальных проблем, исследование метдов их решения, создание прототипов и готовых систем. В зависимости от сферы исследований и задаваемых требований этот список может изменяться, количество необходимых этапов может как увеличиваться так и уменьшаться, иногда даже невозможно заранее оценить все необхдимые этапы из-за высокой сложности исследований. Казалось бы, в общем случае невзможно создать алгоритм, способный определять полную последовательность действий, однако как минимум один компонент, общий для всех исследований всегда присутствует - это результат данного исследования, выраженный в виде документа.

Документ, получаемый в результате работы может быть распространен в неком сообществе для ознакомления и использования его для дальнейших работ, также документ подтверждает наличие результата и позволяет суммировать, сохранить и передать другим плученную информацию. Таким образом создание данного документа не только необходимый при проведении исследования этап, но довольно важный аспект всей работы, что опеределяет требования к его созданию. Поскольку разрабатываемая тестовая система в данный момент рассчитана только на создание документов, соответствующих профилю кафедры, различия в ограничениях на создания данных документов здесь и далее будут фиксированы внутреними уставами кафедры и возможность их изменения предусмотрена не будет, вместо этого, основной целью будет функционал финальной системы.

В общем случае создание документов подразумевает работу со следующими компонентами:

\begin{itemize} 
	\item Форматирование всех остальных блоков: взаимное расположение, структура документа, расположение на странице, ссылки и управление, списки и другие перечисления, статические блоки и тд.
	\item Текстовые объекты с разнообразными возможностями отображения: изменение размера, выделение цветом, различные стили и шрифты, скрытые символы управления.
	\item Вывод и упорядочивание большого количества информации: таблицы, базы данных, графики.
	\item Формулы и другие виды информации о проблемной области.
	\item Библиографическая информация о использованной литературе и её характеристиках
	\item Блоки кода, с возможностью отображения синтаксисов и проверки языковых конструкций.
	\item Изображения, соответствующие данному исследованию или иллюстрирующие некоторые процессы.
	\item Схемы, описывающие определенные явления в проблемной области и упрощающие восприятие документа.
\end{itemize}

Поскольку при создании документов используются вышеперечисленные компоненты в различном соотношении, то в зависимости от предпочтений и требуемого функционала используется достаточно большой и разнообразный набор программного обеспечения. В теории возможно объединить весь необходимый функционал в одно приложение, однако на практике приложение либо получится довольно большим и множество его функций не будут использоваться, либо наоборот, пользователям будет не хватать некоторых функций. Также при использовании такого подхода будт использоваться вычислительные ресурсы устройств, что тоже может стать проблемой.

Исходя из этого следует проанализировать существующие системы, в которых на данный момент создаются документы для учебно-исследовательских работ с учетом их конструктивных возможностей, а также накладываемых ограничений.

Начнем, пожалуй, с самого распространенного ПО для редактирования документов - Microsoft Office Word. Имея широкое распространение среди пользовательсих систем данное ПО приобрело высокую популярность и для создания исследовательских документов. Разнообразный функционал, интеграция под множество платформ и распространенность быстро вытеснить более простые редакторы текстовых документов.Однако за внешним удобством данного ПО скрывается определенное количество проблем, которые, в свою очередь, затрудняют создание документов. Во-первых, межплатформенная интергация продукта получила большую зависимость от конкретных систем, а это приводит к тому, что один и тот же документ, открытый на разных устройствах может иметь разное форматирование и выглядеть по-разному, из-за чего передача  документов может стать затруднительной ввиду наличия дополнительной работы по проверке идентичности. Во-вторых из-за наличия сложных правил форматирования документа многие простые действия пользователя могут привести к нелогичным результатам и время, затраченное на поиск правильного подхода к искомому действию зачастую может занимать довольно большое количество времени. В-третьих работа с документами на пользовательских устройствах не только использует их вычислительые ресурсы, но и зачастую из-за невнимательности пользователя может создать утечку данных из-за которой пользователи , не имеющие доступа к документу, могут получить его.

Далее рассмотрим LaTeX-системы для создания документации. Поскольку данные системы требуют специфических знаний от пользователей для работы с ними, поэтому данный вид систем получил распространение лишь в ограниченных кругах и в общем случае неизвестен обычному пользователю.

\section{Анализ методов обеспечения безопасности веб-приложений}

\section{Методы авторизации в приложении}

\section{Наполнение профиля пользователя}




\section{Обзор результатов предыдущей работы}

Созданный ранее прототип системы визуализации трехмерных сцен на основе трассировки лучей улучшил свою алгоритмическую базу и функциональную составляющую по сравнению с его первой версией, однако его улучшение - это постоянная необходимость. Одной из основных проблем на данном этапе разработки становится то, что постепенное расширение функционала прототипа привело к усложнению его базовых структур и алгоритмов, что затрудняет работу с ним. Хотя улучшение работы с пользователем посредством создания интерфейса было приведено в предыдущей работе, остается еще одна большая часть, в которой заметны проблемы - а именно взаимодействие с разработчиком. Особенно это заметно автору статьи при разработке новых модулей. 

Это возникает из-за множества факторов, в том числе:


\begin{enumerate} 
	\item Разногласия в модулях, на которых основан прототип. Это возникает из-за  наложения множества ошибок, неучтенных их разработчиками и, к сожалению не может быть исправленно немедленно, т.к. данные ошиби находятся на рассмотрении, а изменение основных библиотек на иные потребует большого количества времени.
	\item Некоторые модули, а именно графические шейдеры, необходимые для получения конечного  изображения требуют для понимания комплексные знания о трехмерной графике и логике шейдеров, что является затруднительным для понимания сторонним разработчиком.
	\item Различия в стилистике кода модулей, соответствующих различным прототипам. Так как процесс разработки прототипа занимал продолжительное время и прерывался на получение новых знаний в области, а также исследования новых методов, то различные модули разрабатывались в разных условиях и в разное время, что делает их не похожими друг на друга по стилистике. В дополнении к этому покрытие кода комментариями также неполное.
	\item Большие различие между стандартными библиотеками визуализации и методом тарссировки лучей, что определет высокий порог вхождения для сторонних разработчиков.
	\item Отсутствие системы для "легкого" прототипирования новых возможностей и алгоритмов, обусловленное нестандартностью подходов к разработке и необходимости соблюдать дополнительные правила, такие как выравнивание по памяти и низкоуровневая оптимизация. Также для работы новые модули должны поддерживать текущий функционал, что само по себе является сложной задачей.
\end{enumerate}


Помимо этого следует продолжать исследования совместимости системы частиц и алгоритма трассировки лучей для нахождения новых методов их взаимодействия, их последующего изучения, разработки и поиска возможностей применения.

Таким образом для того, чтобы проводить дальнейшую разработку следует добавить новые методы взаимодействия разработчиков с прототипом, организовать структуру модулей программы для лучшего их понимания, увеличить количество комментариев в коде, а также провести исследование о новых способах взаимодействия между системой частиц и алгоритмом трассировки лучей.

\section{Расширение возможностей разработки прототипа}

Расширение возможностей самого прототипа привело к довольно важной проблеме: без использования стандартизации и с ограничениями в доступе к инструментам разработки сложность разработки проекта растет быстрее чем его функционал. Это наталкивает на мысль о необходимости переосмысления подхода к разрботке.

Первое из возможных решений, приходящих в голову - введение жесткой стандартизации на разрабатываемые модули, сопровождение их множественной документацией и тщательная проработка концепта перед реализацией. Однако условия в виде ограниченного времени, необходимости проведения исследований и общий вид проекта как "доказательство концепции" в большинстве своем не позволяют применять этот метод. Это объясняется тем, что тщательная проверка каждого предположения перед непосредственной разработкой превысит допустимое время, выделенное на проект, а исследовательская составляющая зачастую требует переработки модулей, уже считавшихся завершенными, что потребует в разы больше работы, чем их непосредственная проверка при разработке. 

Следующее, и куда более применимое решение - создание дополнительного инструментария для облегчения разработки. Этот вариант лучше вписывается в концепцию прототипа алгоритма трассировки лучей, так как позволяет не изменять уже созданные модули, при этом обеспечивает возможность последующего тестирования новых концептов и алгоритмов при меньших затратах, а также позволит уменьшить порог вхождения в проект для сторонних разработчиков.

Приняв этот вариант как рабочий следует продумать его основные аспекты:

\begin{itemize}
	\item Совместимость с текущей реализацией является необходимым требованием, так как любые изменения в текущей системе довольно затратны.
	\item Понятность работы с новым инструментом должна обуславливаться использованием стандартных реализаций для уменьшения входного порога в проект.
	\item Функциональность инструмента должна соответствовать сложности его разработки, иначе создание данного инструмента не ускорит общую разработку.
	\item При разработке необходимо использовать минимальное количество сторонних зависимостей во избежании повторения ошибок их объединения.
	\item Использование зависимостей на текущую реализацию также должно быть минимальным во избежании повторения реализации.
	\item Наличие возможности добавления функционала инструмента по частям в зависимости от необходимости.
	\item Инструмент должен позволять избавляться хотя бы от одного минуса прототипа. 
	\item Предпочтение отдается простым в управлении и подключении решениям.
\end {itemize}

Объединив данные критерии можно предложить общую концепцию инструмента: создание альтернативной системы визуализации, поддерживающей структуры типов, используемые в текущей реализации, а также способной работать с частью реализованных алгоритмов (оптимально - со всеми, кроме непосредственно алгоритма трассировки лучей), при этом использующую для своей реализации одну из стандартных технологий визуализации. 

Данный инструмент позволит тестировать прототипы внутренних алгоритмов, при этом работая намного быстрее текущего прототипа. Также он позволит проводить большее количество тестов, сохраняя возможность пользоваться интерфейсом и использовать визуальное представление как результат визуализации, что труднодостижимо в текущем прототипе. Использование стандартных методов позволит ускорить разработку, так как многие концепты уже реализованы в существующих библиотеках, а также улучшит узнаваемость кода для сторонних разработчиков. А возможность использования стандартной технологии визуализции только в тех аспектах, которые необходимы, позволит ускорить разработку, при этом оставляя возможность дальнейшей модернизации. И последним улучшением, привнсимым данным инструментом станет возможность сравнения результатов визуализации стандартной технологии и  визуализации посредством системы частиц и ви зуализации преимуществ последней.

\section{Анализ улучшения взаимодействия системы частиц и алгоритма трассировки}

Текущая версия прототипа системы визуализации частиц связывает её с системой частиц посредством непосредственной визуализации частиц методом трассировки лучей, однако стоит изучить расширенное взаимодействие. Для этого необходимо выявить в системах концепты которые бы сочиталис друг с другом, тем самым улучшая прототип в целом, при этом существует возможность их реализации и совместимость с текущей системой.

Один из рассмотренных вариантов - подход к обоим частям прототипа с позиции совпадения и  структуры. В таком случае были обнаружены незначительные совпадения, из которых самым значимым было испльзование координат как для хранения даннох о частицах, так и для рассчетов лучей. В обоих системах используется прямоугольная система координат. Однако стоит рассмотреть и применение альтернативных систем координат. Из-за своей специфичности и сложности в рассчетах были отброшены неевклидовые системы координат. Для рассмотрения остальных был введен критерий - так как основное значение, вычисляемое в процессе является расстояние, то оптимальней всего  подходит сферическая система координат.

Давайте рассмотрим переход на сферические координаты поподробней. Имея преобразование из прямоугольных в сферические коррдинаты как систему уравнений - мы подтверждаем, что часть прототипа, работающая с частицами практически не изменится. Однако использование сферических координат в алгоритме трассировки лучей довольно эффективно: сферические координаты представляют собой вектор с 2 углами направления и модулем расстояния, что и является необходимым нам лучом, направленным на конкретный объект. Несмотря на это, проблема поиска пересечения луча и объекта, являющаяся одной из самых затратных в алгоритме не решается с помощью простого перехода на сферические координаты.

Рассмотрим рисунок (\ref{pic:B11}. На нем есть 2 частицы: зеленая и синяя, обозначенные кругами соответсвующих цветов и обладающие радиус-вектором положения в сферических координатах а также собственным радиусом. Предположим что луч, созданный алгоритмом трассировки совпал по направлению с радиус-вектором синей частицы. В таком случае если нет частиц, обладающих таким же радиус вектором, но с меньшим расстоянием, алгоритм должен отрисовать данную частицу. Однако в данном случае зеленая частица, находящаяся центром вне обзора камеры, все равно должна быть отрисована и при этом синяя частица не будет отрисована. Таким образом данное решение не избавляет нас от задачи поиска. Возможно частичное применение данного метода, при предположении о близком к нулю собственному радиусу всех частиц, однако в таком случае придется использовать аппроксимацию попадания лучей, что может привести к некорректному конечному изображению.


\begin{figure} 
\begin{center}
\includegraphics[width=.5\columnwidth]{./img/B11.png}
\end{center}
\caption{Проблема сферических координат.}
\label{pic:B11}
\end{figure}

Исходя из этого стоит сделать вывод, что данный метод позволяет улучшить взщаимодействие лишь в частичном случае, что может быть полезно, однако н подходит для текущего прототипа.

Используя иной подход к поиску новых метдов взаимодействия, стоит отталкиваться не от принципов, лежащих в основе обеих систем, а от их семантической состовляющей. Система частиц позволяет моделировать объекты как соввокупность частиц, когда алгоритм трассировки лучей позволяет визуализировать эти объекты с помощью исследования их пересечения с множеством лучей. Общее в их семантике  - их зависимость от объекта. Таким образом, модифицируя понятие и представления объекта мы можем получить новые методы взаимодействия. 

К сожалению разработка основанная на введении эмпирических понятий довольна сложна и не подходит под концепт текущего прототипа системы визуализации. Тогда единственная область в которой возможны дальнейшие поиски - это методы представления объектов. Так как прототип работает в трехмерном пространстве, то и объекты соответственно должны быть трехмерными. А учитывая то, что за взаимодействие внутри объекта уже отвечает система частиц, то выбор сужается до работы с повехностями трехмерных объектов. Подробнее об этом в следующем разделе.

\section{Методы работы с трехмерными поверхностями}

Современные методы работы с трехмерными поверхностями предоставляют большой выбор технологий разработки, зависящий от множества критериев, предъявляемых к результату а также назначений использования данного результата. Из-за схожести методов их можно разделить на группы и оценить каждую на применимость к текущему прототипу.

Начать стоит с полигонального моделирования. Это один из самых распространненых способов представления объектов в системах визуализации. Полигональное моделирование дает возможность производить различные манипуляции с сеткой трехмерного объекта на уровне подобъектов: вершин, ребер, граней. Сам полигон состоит из граней, но в системах, которые поддерживают многосторонние грани, полигоны и грани будут равнозначны.  Это основной вид моделирования, так как при помощи его можно создать объект любой сложности путем соединения групп полигонов. Однако наравне с преимуществами можно выделить и недостатки данного метода: объем, необходимый для хранения памяти объекта растет соизмеримо его сложности, для получения более детальных результатов необходимо увеличивать количество полигонов, зачастую могут появляться выступающие ребра, искажающие объект, устранение которых требует дополнительной обработки всей модели. Пример полигональной модели приведен на рисунке  (\ref{pic:B12}).

\begin{figure} 
\begin{center}
\includegraphics[width=.5\columnwidth]{./img/B12.png}
\end{center}
\caption{Пример полигональной сферы.}
\label{pic:B12}
\end{figure}

Следующим рассмотреным методом является 3d-скульптинг. Он представляет собой имитацию процесса «лепки» трехмерной модели, то есть деформирование её полигональной сетки специальными инструментами – кистями. Можно провести аналогию с лепкой фигур руками из пластилина или глины. При этом явно выделяется существенный минус - необходимость обработки модели человеком, так как программное описание данного метода на данный момент нереализуемо. Пример скульптуры из стандартного набора моделей Blender можно увидеть на рисунке (\ref{pic:B13}).

\begin{figure} 
\begin{center}
\includegraphics[width=.5\columnwidth]{./img/B13.png}
\end{center}
\caption{Пример трехмерной скульптуры.}
\label{pic:B13}
\end{figure}

Очередным возможным методом является твердотельное моделирование. Если при полигональном моделировании куб разрезать пополам, то внутри будет пустота. При твердотельном моделировании, если разрезать куб, то  пустоты не будет, как если бы разрезали реальный твердый предмет. В твердотельном моделировании при построении модели работают сразу со всей оболочкой, а не с отдельными поверхностями. Сначала создается простая форма оболочки, например, сферы, а затем к ней применяют различные операции: резка, объединение с другими телами, булевые операции и др. Этот метод позволяет представлять модели точнее, однако сильно зависит от заранее заданых примитивов в виде начальных твердых тел, а также требует либо ольшие вычислительные мощности для вычислиения пересечения объектов, либо использование приближений при вычислении, что ведет к повышению неточности. 

Уже известное из предыдущих работ параметрическое моделирование. Оно осуществляется путем введения требуемых параметров элементов модели, а так же соотношение между ними. Иными словами создается математическая модель с нужными параметрами, изменяя которые можно создать различные комбинации модели и тем самым избежать ошибок, внеся необходимые корректировки. Минусы также известны - сильная зависимость модели от начального базиса формул, математическая сложность описания некоторых объектов.

Один из новых способов - моделирование метасферами. Его особенность в том, что модель строится из трехмерных объектов сглаженной замкнутой формы (метасфер), которые при соприкосновении друг с другом автоматически сливаются частями соприкасающихся поверхностей. Метасферы как бы притягиваются друг к другу подобно каплям воды или ртути.Логичные недостатки - сложность рассчета слияния сфер и возникновение неоднозначностей при мделировании. Также стоит заметить что при работе с некоторыми наборами объектов данному сособу потребуется в разы больше памяти для хранения данных, чем другим. Пример моделей, основанных на метаферах можно увидеь на рисунке (\ref{pic:B14}).

\begin{figure} 
\begin{center}
\includegraphics[width=.5\columnwidth]{./img/B14.png}
\end{center}
\caption{Пример моделирования метасферами.}
\label{pic:B14}
\end{figure}

И последним рассмотренным методом будет сплайновое моделирование. Оно представляет собой создание трехмерных объектов при помощи кривых линий (сплайнов). Сплайнами могут выступать линии различной формы: окружности, прямоугольники, дуги и т.д. Объекты при этом получаются плавной формы, в связи с чем, данный метод получил широкое применение в создании органический моделей. Преимущество данного метода в гибкости изменения формы сплайна. Данный вид моделирования часто сравнивают с полигональным, как векторную графику с растровой. Преимущество векторной графики в том, что при увеличении объекта, его качество не изменяется, в отличие от растрового, где становятся видны пиксели. Так же и при увеличении объекта, созданного сплайнами, его качество останется неизменным. При более тщательном рассмотрении данного метода можно заметить что по недотаткам он не превосходит полигональное моделирование, однако несмотря на это он все равно редко используется в системах. Проблема заключается в небольшом развитии данного метода, его низкой популярности в кругах разработчиков, а также повышенной по сравнению с другими методами алгоритмичекой сложностью. Пример модели, основанной на сплайнах можно увидеть на рисунке (\ref{pic:B15}).

\begin{figure} 
\begin{center}
\includegraphics[width=.5\columnwidth]{./img/B15.png}
\end{center}
\caption{Пример моделирования сплайнами.}
\label{pic:B15}
\end{figure}

Таким образом основываясь на возможностях решения определенных задач (написание усиленных агоритмов, организация хранения памяти, оптимизация по скорости и др.) стоит выбирать определенную модель и разрабатывать дальнейшие взаимодействия на её основе. 

\section{Выводы}

Основываясь на предыдущих разделах, можно составить следующие выводы.

\begin{enumerate}
	\item Выполнен обзор предыдущей работы, выявленны основные недостатки и проанализированны основные способи их решения.
	\item Определены возможности и направления разработки инструментов для поддержки разработки и облегчения взаимодействия со сторонними разработчиками.
	\item Изучен методы работы с альтернативными системами координат и обоснована возможность их применения в прототипе, а также рассмотренны недостатки, которые могут возникнуть при работе с ними.
	\item Как возможный путь далнейшего развития взаимодействия системы частиц и алгоритма трассировки лучей рассмотрены их части, найдены общие зависимости и приведено их исследование на возможность замены альтернативными зависимостями.
	\item Рассмотрены основные методы трехмерного моделирования, выявлены достоинства и недостатки, проведен анализ применимости технологий в текущем прототипе, определен метод выбора используемой методологии.
\end{enumerate}



\section{Постановка задачи курсового проекта}

На основе анализа, проведенного в предыдущих подразделах, возможно сформулировать цель, которая должна быть достигнута в рамках данной учебно-исследовательской работы. Необходимо также выделить основные задачи, которые необходимо выполнить для достижения поставленной цели. Целью данной учебно-исследовательской работы является улучшение прототипа системы визуализации трехмерных сцен на основе алгоритма трассировки лучей для проведения последующих исследований, а также интеграции его в систему DHCN.

Помимо выполнения основной цели необходимо также уделить внимание следующим задачам:

\begin{itemize}
	\item Максимизировать показатель FPS системы, используя программные методы.
	\item Создать дополнительный инструментарий для последующей работы с прототипом.
	\item Применить новый инструментарий к текущей системе при минимальном её изменении.
	\item Выбрать и описать новую модель трехмерных объектов.
	\item Провести анализ и разработку новых алгоритмов для работы по выбранной методологии моделирования трехмерных объектов.
\end{itemize}

Выполнение этих задач не обязательно, но желательно для актуализации данного алгоритма.
