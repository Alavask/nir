\chapter*{Заключение}
\addcontentsline{toc}{chapter}{Заключение}

В данной учебно-исследовательской работе был решен ряд задач, результатом выполнения
которых является инструмент визуализации стандартными методами DirectX, интегрированный в систему, а также некоторые основыне алгоритмы, необходимые для реализации метода мделирования объектов на основе сплайнов.

Был проведен анализ предыдущей работы, найдены недостатки и предложены пути решения. Для обеспечения возможности тестирования разрабатываемых алгоритмов без необходимости их адаптации к системе частиц, а также для удобного взаимодействия со сторонними разработчиками был предлжен инструмент визуализации, основанный на стандартны метода DirectX.

Для разработки программного обеспечения были выявлены основные зависимости, сохраняющиеся при переходе от системы частиц к алгоритму трассировки лучей. Это привело к выявлению новых возможностей для улучшения взаимодействия в виде разработки новых методов моделирования объектов. Как самый эффективный для использования в данной системе был выбран метод моделирвания посредством сплйнов.

На стадии проектирования был описан метод разработки инструмента альтернативной визуализации, спроектирован его нтерфейс и определены возмжности. Также было проведено обеспечение совместимости.

В результате реализации был получен инструмент, позволяющий визуализировать сцену без необходимости использования алгоритмов трассировки лучей, а также созданы и протестированы алгоритмы разбиения пространства и поиска по разбиению, которые будут использоваться методом моделирования посредством сплайнов в последующих работах.

Несмотря на то, что разработанный модуль отвечает целям работы, некоторые подзадачи в данной работе достигнуты не были. Их достижение возможно при последующей разработке, в которой упор будет стоять на: оптимизации работы системы, проведении исследований в виде новых фильтров конпечного изображения использующих дополнительные возвращаемые данные, а также возможности использования данного прототипа в реальных условиях.